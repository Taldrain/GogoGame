\documentclass[12pt,a4paper,titlepage]{book}
\usepackage[utf8x]{inputenc}
\usepackage{ucs}
\usepackage[english]{babel}
\usepackage{amsmath}
\usepackage{amsfonts}
\usepackage{amssymb}
\usepackage{makeidx}
\usepackage{graphicx}

\usepackage{eurosym}
\usepackage{fancyhdr,extramarks}
\usepackage{titletoc}

\setlength{\parskip}{1ex} %retour à la ligne
\setlength{\headsep}{40pt}
\setlength{\footskip}{60pt}
\usepackage{setspace}
%\singlespacing
\onehalfspacing
%\doublespacing
%\setstretch{1.1}

%%%%%%%%%%%%%%%%%%%%%%%%%%%%%%%%%%%%%%%%%%%%%%%%%%%%%%%%%%%%%%%%%%%%%%%%%%%%%%%%%%

\newcommand{\HRule}{\rule{\linewidth}{0.5mm}}

\DeclareSymbolFont{extraup}{U}{zavm}{m}{n}
\DeclareMathSymbol{\varheart}{\mathalpha}{extraup}{86}
\DeclareMathSymbol{\vardiamond}{\mathalpha}{extraup}{87}

\titlecontents{chapter}[3pc]{\addvspace{1.5pc}\bfseries\filright}{\contentslabel[\thecontentslabel]{3pc}}{}{\hfill\contentspage}[\addvspace{2pt}]
\makeatletter
\renewcommand\chapter{\clearpage\@startsection{chapter}{0}{1em}{\baselineskip}{2\baselineskip}{\Huge\textbf}}
\makeatother

%En-tête et pieds de pages
\pagestyle{fancyplain}
	\lhead[\emph{\nouppercase{\leftmark}}]{\emph{\textit{AI Report}}}
	\chead{}
	\rhead[\emph{\textit{AI Report}}]{\emph{\nouppercase{\leftmark}}}
	\lfoot[]{\small{\textit{Thomas Wickham}}}
	\rfoot[\small{\textit{Thomas Wickham}}]{}

%%%% macro caption %%%%
\makeatletter
\def\@captype{table}
\makeatother
%%%% fin macro %%%%

%%%%%%%%%%%%%%%%%%%%%%%%%%%%%%%%%%%%%%%%%%%%%%%%%%%%%%%%%%%%%%%%%%%%%%%%%%%%%%%%%%


\author{WICKHAM Thomas}

\title{AI Report}

\makeindex

\begin{document}

\begin{titlepage}

\begin{center}
\textsc{\LARGE AI Report}\\[0.5cm]


% Title
\HRule \\[1.0cm]
{\Huge \bfseries Go game (Wei Qi)\\[2ex]
Artificial Intelligence}\\[1.0cm]

\HRule \\[2.0cm]

% Author and supervisor
\begin{minipage}{0.4\textwidth}
\begin{flushleft} \large
\emph{Author:}\\
Thomas \textsc{Wickham}
\end{flushleft}
\end{minipage}
\begin{minipage}{0.4\textwidth}
\begin{flushright} \large
\emph{Instructor:} \\
Pr. Liqing \textsc{Zhang}
\end{flushright}
\end{minipage}

\vfill

% Bottom of the page
{\large February 22th 2012}

\end{center}

\end{titlepage}

\newpage


\tableofcontents

\chapter*{Introduction}

This report is about the GO contest of the AI course.\\
It contains detailed description of the AI project.\\
\\
At the beginning the project was designed in C++ but then we decide to change to
a completly different langage OCaml.\\
With this powerfull langage we managed to write some great algorithm like an UCT
or a Neural Network.\\


\chapter{Common Part}

In this part, we will discut about the common work of the group.

Firstly, we will discuss our choice to use OCaml, a functionnal language.
Then, we will see the differents algorithms we choose to implement and, last but not least, we will analyze our results.

\section{The choice of a functionnal Language}

\subsection{Explanations}

\subsection{Why OCaml}

\section{Algortihms}
\subsection{Upper bound for Confidence Tree (\textsc{Uct})}

\subsection{Negascout}
\subsubsection*{Groups}
In this game, the most important thing is, of course, the stones.\\ 
During the game, the stones are added one per one. So that, some
groups of stones can be finally created. The better groups you own, 
the better chances to win you have. The algorithm Negascout have to 
find the way to make the best groups.\\
\\
First at all, what's exactly a group in our porject? It's a set which owns 2 labels :
the liberty of the group which is the number of empty cases around it and 
the list of the stones which are the group itself. So, let's talk about 
how do we build this structure.\\
When a stone is added to the board, all the cases around this stone are checked.
We don't refresh all the board but only the stones in contact with the stone 
that we played. By the way, there is 3 differents cases.\\ 
\\
If the case is empty, the group of this stone will have one more liberty.
If there is a already a stone, depend of the color. When it is the same color as
the added stone, they will be in the same group and the liberty of this new group
depends of the liberty of the stone that is why we have to check the stone and
add its group if needed. Last case is when there is an opposite color as yours,
just decrease the liberty of the other group that can detroy it if the is no 
liberties left.\\
\\
That is why we implemented and use groups. It is the better way to manages stones.\\

\subsection{Genetic Algorithm and Neural Network}
Because we knowed that a lot of game of Go (already played) are easily
accessible we wanted to have an easy way to know if it's better to attack or
defend at a t time. This is possible by a genetic algorithm and a neural
network.\\
\\
We create a neural network, no intern layout it's not necessary for what we
want, that have in input the number of vertice in a Go board and in output we
have the response, we know if we have to attack or defend.\\
Of course it's necessay to do some learning because otherwise we would have to
test each combination of weight for each node. Speaking of which, the function
used to determine the output is the Sigmoid function.\\
\\
So, the other part is the genetic algorithm. It's independant with the
neural network but necessary. In our project we used it to shuffle, mutate and
do some cross-over with our data of weight.\\
After that we have a completly new population, supposed better, that we take to
replace our current data of weight in our neural network.\\
\\
These two algorithms have been implemented in a completly generic way so that we
can use them for whatever we want.\\

\section{Results}

\chapter{Individual Part}

\section{Tasks}

\section{Personnal value in the project}

\section{Individual Conclusion}

\chapter*{Conclusion}
plap

\end{document}
